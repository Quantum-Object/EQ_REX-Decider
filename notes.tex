
## TASKS:
1.a- w to NFA,
1.b- print NFA in a nice way,
2- L(NFA)={} Decider (E_NFA) ,
------

3- visualiser of NFA ....

---

## IMPORTANT TASKS:
1- post prefix thing 
2- more coding
3- hard part





an NFA is :
1 a list of integers Q: states
2 map (int,list(int)) this should work as f (function for transition)
3 list A (accepted states)
------
HOW TO turn a Rex to a NFA 
(1|0)*(1*)


------

functions:
1.union : 
on input m:
    
2.not   :
3.inter : use union and not (that is inter(a,b)=not(union(not a,not  b)))
IDK about time complexity but it is not O(2^n)
4.E_NFA(NFA X)


now if we have NFA1 and NFA2  we can get 
to get the (A^not(B)) U (not(A) ^ B)= C
so this is as follows:
union (inter(a,not b),inter (b,not a))

OK now :




""" 
(X)->X
X+Y-> Conc(X,Y)
ua*b-> Conc(u,conc(a.S(),b))
X|Y-> union(X,Y)

""" # (*,+,|) and (0,1) or (a,b,c)
# so now we should look for brackets and strat parsing them in some stack manner
# for instance
#1. break s into ()()()().... 
#2. make NFA for each of these:
#3. so now make subNFA OK
# a string is 


    
    

    



# what can we have now with no brackets well:
# like (1*0 or 1+1+0 clearly we have to take it in account 1111==1+1+1+1 
#now priorty of operation as follows:
# star,conc,union (*,+,|)
# for r1*r2+r3|r4
# this means we should do it recursively starting from | and then + then *


we need to finish :
1-RPN (postfix)
2-evalute it aka toNFA

3- simulate(w,NFA) !!